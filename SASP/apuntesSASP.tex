% ============================================================================
% DOCUMENTO: APUNTES DE SOSTENIBILIDAD APLICADA AL SISTEMA PRODUCTIVO
% ============================================================================
% Descripción: Apuntes académicos estructurados en partes, unidades y secciones
%              sobre sostenibilidad en entornos empresariales y profesionales
% Autor:       Francisco IndexOutOfBounds
% Curso:       2025/2026
% Versión:     1.0
% ============================================================================

% ============================================================================
% CONFIGURACIÓN DEL DOCUMENTO
% ============================================================================
% Clase: book (documento tipo libro)
% - 11pt: Tamaño de fuente base (11 puntos)
% - titlepage: Crea una página de título personalizada
\documentclass[11pt,titlepage]{book}

% ============================================================================
% PAQUETES MATEMÁTICOS Y SÍMBOLOS
% ============================================================================
% amsmath: Entornos mejorados para ecuaciones matemáticas
% amssymb: Símbolos matemáticos adicionales (American Mathematical Society)
% amsfonts: Fuentes matemáticas especiales
% latexsym: Símbolos LaTeX adicionales
\usepackage{amsmath,amssymb,amsfonts,latexsym}

% ============================================================================
% CODIFICACIÓN Y MULTIIDIOMA
% ============================================================================
% Codificación UTF-8: Permite usar caracteres acentuados y especiales directamente
\usepackage[utf8]{inputenc}
% Soporte para español e inglés: Define reglas de idiomas para separación de sílabas
% y formato de texto según el idioma
\usepackage[english,spanish,es-tabla]{babel}

% ============================================================================
% GRÁFICOS E IMÁGENES
% ============================================================================
% Paquete para incrustar imágenes en formato JPG, PNG, PDF, etc.
\usepackage{graphicx}

% ============================================================================
% HIPERVÍNCULOS Y REFERENCIAS
% ============================================================================
% Habilita hipervínculos interactivos en el documento PDF
\usepackage{hyperref}
% Configuración personalizada de hipervínculos:
\hypersetup{
    colorlinks=true,       % Activar colores en enlaces (en lugar de cajas)
    linkcolor=blue,        % Color azul para referencias internas (\ref, \cite)
    filecolor=magenta,     % Color magenta para enlaces a archivos externos
    urlcolor=cyan,         % Color cian para URLs web
    pdftitle={Sostenibilidad Aplicada al Sistema Productivo}, % Metadato: título PDF
    pdfpagemode=FullScreen, % Mostrar el PDF en modo pantalla completa al abrirlo
}
% Usar el mismo estilo de fuente para URLs que el resto del texto
\urlstyle{same}

% ============================================================================
% CONFIGURACIÓN DE MÁRGENES Y LAYOUT
% ============================================================================
% Paquete geometry: Control preciso de dimensiones y márgenes de página
\usepackage{geometry}
% Configuración:
% - a4paper: Tamaño de papel A4 (210x297 mm)
% - left=30mm, right=25mm: Márgenes laterales
% - top=30mm, bottom=30mm: Márgenes superior e inferior
\geometry{a4paper, left=30mm, right=25mm, top=30mm, bottom=30mm}

% ============================================================================
% ENCABEZADOS Y PIES DE PÁGINA
% ============================================================================
% fancyhdr: Permite personalizar encabezados y pies de página
% fancybox: Proporciona cajas decorativas y marcos para texto
\usepackage{fancyhdr}
\usepackage{fancybox}

% ============================================================================
% ESTILOS DE CAPÍTULOS Y SECCIONES
% ============================================================================
% fncychap: Estilos decorativos para títulos de capítulos
% [Bjornstrup]: Estilo elegante con líneas y formateo especial
\usepackage[Bjornstrup]{fncychap}

% titlesec: Control avanzado sobre el formato de títulos de secciones
% y subsecciones (espaciado, color, numeración, etc.)
\usepackage{titlesec}

% ============================================================================
% CONFIGURACIÓN DE ESTILOS Y FORMATEO
% ============================================================================

% Eliminar indentación automática de párrafos (primera línea)
% Cuando \parindent=0pt, los párrafos comienzan sin espaciado a la izquierda
\setlength{\parindent}{0pt}

% ============================================================================
% FORMATO DE PARTES (TRIMESTRES)
% ============================================================================
% Define el aspecto visual de los títulos de partes (trimestres del curso)
% [frame]: Encierra el título en un recuadro
% \filright\footnotesize\enspace\bfseries TRIMESTRE \thepart\enspace:
%   - \filright: Alinea a la derecha
%   - \footnotesize: Tamaño de fuente pequeño
%   - \bfseries: Texto en negrita
%   - "TRIMESTRE X": Prefijo del número de parte
\titleformat{\part}[frame]{\normalfont}{\filright\footnotesize\enspace\bfseries TRIMESTRE \thepart\enspace}{8pt}{\Large\bfseries\filcenter}

% ============================================================================
% FORMATO DE SECCIONES (UNIDADES DENTRO DE PARTES)
% ============================================================================
% Cambia la numeración de secciones a números simples (1, 2, 3...) 
% en lugar de la numeración jerárquica (1.1, 1.2...)
\renewcommand{\thesection}{\arabic{section}}
% Formato personalizado:
% [block]: Título en bloque (ocupando todo el ancho)
% {\doublebox{\thesection}}: Número de sección con doble recuadro
% .5em: Espacio entre número y título
\titleformat{\section}[block]{\normalfont\bfseries}{\doublebox{\thesection}}{.5em}{}[\titlerule\vspace*{4pt}]

% ============================================================================
% FORMATO DE SUBSECCIONES Y SUBSUBSECCIONES
% ============================================================================
% Subsecciones: Alineadas a la derecha con formato en negrita y grande
% [hang]: El título cuelga bajo el número (no en la misma línea)
\titleformat{\subsection}[hang]{\bf\large\raggedright}{\filleft\thesubsection.}{1 em}{}

% Subsubsecciones: Alineadas a la izquierda con formato similar
\titleformat{\subsubsection}[hang]{\bf\large\raggedleft}{\thesubsubsection}{0em}{}

% ============================================================================
% INICIO DEL DOCUMENTO
% ============================================================================

% ============================================================================
% INICIO DEL DOCUMENTO PRINCIPAL
% ============================================================================
% Todo el contenido se coloca entre \begin{document} y \end{document}

\begin{document}

% ============================================================================
% PÁGINA DE TÍTULO PERSONALIZADA
% ============================================================================
% Crea una portada atractiva con:
% - Título principal del documento
% - Imagen de portada (portada.jpg)
% - Nombre del autor
% - Información del curso
\begin{titlepage}
  % Centra el contenido horizontalmente
  \centering
  
  % Título pequeño
  {\LARGE \textsc{Apuntes}\par}
  \vspace{1em}
  
  % Título principal en negrita
  {\huge\bfseries Sostenibilidad Aplicada al Sistema Productivo\par}
  \vfill
  
  % Imagen de portada escalada al ancho total de la página
  \includegraphics[width=\textwidth]{img/portada.jpg}
  \vfill

  % Información del autor y curso al final
  {\Large Francisco IndexOutOfBounds\par}
  \vspace{0.5em}
  \textsc{Curso 2025/2026}
\end{titlepage}

% ============================================================================
% TABLA DE CONTENIDOS
% ============================================================================
% Genera automáticamente un índice con todos los capítulos, secciones y subsecciones
% Los números de página se vinculan a las respectivas secciones en el PDF
\tableofcontents

% ============================================================================
% CONTENIDO PRINCIPAL DEL DOCUMENTO
% ============================================================================
% El contenido se organiza en partes (trimestres) y unidades

% PRIMER TRIMESTRE: SOSTENIBILIDAD EN EL ENTORNO EMPRESARIAL Y PROFESIONAL
% ============================================================================
% Define la primera parte del documento
% El contenido será numerado automáticamente como "TRIMESTRE 1"
\part{Sostenibilidad en el entorno empresarial y profesional}

% Inclusión de archivos LaTeX externos:
% \input{ruta/archivo.tex} inserta el contenido del archivo en esta posición
% Las unidades 1 y 2 corresponden al primer trimestre
\chapter{Sostenibilidad en las organizaciones empresariales}
\lipsum[1]

\section{Sostenibilidad y marco jurídico}
\lipsum[1]
    \subsection{Concepto de sostenibilidad}
    \lipsum[1]
    \subsection{Marco jurídico}
    \lipsum[1]
        \subsubsection*{Agenda 2030}
        \lipsum[1]
        \subsubsection*{Marco de Cooperación de las Naciones Unidas para el Desarrollo Sostenible}
        \lipsum[1]
        \subsubsection*{Acuerdo de París}
        \lipsum[1]
        \subsubsection*{Marco Estratégico de Energía y Clima}
        \lipsum[1]

\section{Agenda 2030 y los ODS}
\lipsum[1]
    \subsection{¿Qué son?}
    \lipsum[1]
    \subsection{Las 5P del desarrollo sostenible}
    \lipsum[1]

\section{Aspectos ambientales, sociales y de gobernanza (ASG)}
\lipsum[1]
    \subsection{Criterios ASG}
    \lipsum[1]
        \subsubsection*{Aspectos ambientales}
        \lipsum[1]
        \subsubsection*{Aspectos sociales}
        \lipsum[1]
        \subsubsection*{Aspectos de gobernanza}
        \lipsum[1]
    \subsection{Inversores socialmente responsables (ISR)}
    \lipsum[1]
    \subsection{Ventajas de impulsar políticas ASG}
    \lipsum[1]

\section{Stakeholders y los criterios ASG}
\lipsum[1]
    \subsection{¿Qué son?}
    \lipsum[1]
    \subsection{Clasificación}
    \lipsum[1]

\section{Medición de la sostenibilidad}
\lipsum[1]
    \subsection{Indicadores clave de desempeño}
    \lipsum[1]
    \subsection{Indicadores de sostenibilidad}
    \lipsum[1]
        \subsubsection*{Indicadores de sostenibilidad ambiental}
        \lipsum[1]
        \subsubsection*{Indicadores de sostenibilidad social}
        \lipsum[1]
        \subsubsection*{Indicadores de sostenibilidad institucional}
        \lipsum[1]
        \subsubsection*{Indicadores de sostenibilidad más eficaces}
        \lipsum[1]

\input{units/unit2.tex}

% SEGUNDO TRIMESTRE
% ============================================================================
\part{}

% Inclusión de la unidad 3 (segundo trimestre)
\input{units/unit3.tex}

% ============================================================================
% FIN DEL DOCUMENTO
% ============================================================================

\end{document}